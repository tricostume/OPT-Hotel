\section{Previous heat and control analysis}
In order to analyse the heat fluxes governing the heat transmission of a building there are two ways to proceed: taking blueprints of the building and, knowing all the material properties, making a heat analysis of every room, or taking the temperature of every room and identify the parameters to know the behave of heat fluxes. Because both approaches lead to the solution, we decided to make two independent previous analysis. \\
Each procedure is based on the equivalence having on the left side the summation of heat fluxes at a specific time instant and on the right side the variation of the internal energy from an instant to the following one:
\begin{align*}
& q_{tot_{t}} = \Sigma q_{k_{t}} + \Sigma q_{h_{t}} + \Sigma q_{vent_{t}} + \Sigma q_{sun_{t}} + q_{pump_{t}} \\
& q_{tot_{t}} = \rho V c_{p} \frac{d T}{d t} \numberthis
\label{eq:model}
\end{align*}


Where:

$q_{k}$ is the thermal convection; \\
$q_{h}$ is the thermal conduction; \\
$q_{vent}$ is the ventilation flux according to the norm UNI/TS 11300; \\
$q_{sun}$ is the sun radiation took from weather forecast; \\
$q_{pump}$ is the heat flux coming from the heat pump (if activated); \\
$\rho V c_{p} \frac{d T}{d t}$ is the time derivative of the internal energy.

\subsection{Blueprint analysis strategy}
Since the used scenarios must be realistic we planned three kind of room according with their price and structure ($25$ $m^2$, $50$ $m^2$, $75$ $m^2$) and the same amount of customer kinds.
Every wall is made by common bricks (density: $2000$ $\frac{kg}{m^3}$; heat capacity: $0.9$ $\frac{kJ}{kg K}$; thermal conduction: $8 e^{-4}$ $\frac{kJ}{s m K}$); the exterior wall thickness is $0.4$ $m$, while the interior one is $0.1$ $m$. Every room have at least a window and one door on the corridor. The windows are made by common glass (density: $2400$ $\frac{kg}{m^3}$; heat capacity: $0.84$ $\frac{kJ}{kg K}$; thermal conduction: $9.6 e^{-4}$ $\frac{kJ}{s m K}$), and its thickness is $0.04$ $m$, while the doors are assumed to be like the interior wall for their heat behaviour. To run these initial set of experiments we decided to use real heat pumps (any other kind of heat source do not change the result) taken by the commercial catalogue Daikin Industries, in particular we used the heat pump called 'FTXZ35N'.

Due to the fact computer cannot handle continuous domain, the heat transfer equation must be discrete:
$$q_{tot_{t}} = \rho V c_{p} \frac{d T}{d t} \hspace{5mm} --> \hspace{5mm} \frac{q_{tot_{t}}}{\rho V c_{p}} = \frac{T_{t+1} - T_{t}}{time \\ gap}$$
Then, we can rewrite the equation:
$$K_{i,j} (T_{i,t} - T_{j,t}) = \frac{C_{i}}{time \\ gap} (T_{i,t+1} - T_{i,t})$$
Where:

$K_{i,j}$ is the heat transfer coefficient from the room j to i; \\
$C_{i,j}$ is the capacitance of the room i;

\subsection{System identification strategy}
The practise requires the temperature of the air of each room at each instant of the analysis   



E-plus

bla bla

\subsection{P control vs. ON/OFF control}


map of the hotel

different curves for the temperature depending on floors

gap-time

Assumptions on the hotel (temperature in - out, sun)


